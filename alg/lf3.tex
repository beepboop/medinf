% \documentclass{fp-slides}
\documentclass[10pt,portrait]{beamer}
\usepackage{mathtext}
\usetheme{Warsaw}
\newcommand{\maybepause}{}
%\newcommand{\maybepause}{\pause}
\setlength{\floatsep}{8pt plus 2pt minus 2pt}
\setlength{\textfloatsep}{8pt plus 2pt minus 2pt}
\setlength{\intextsep}{12pt plus 2pt minus 2pt}
\AtBeginDocument{%
% \selectlanguage{russian}%
\frenchspacing
\righthyphenmin=2
\sloppy
%\author{Shiray Andrey}
\institute{O.O.Bogomolets National Medical University}
\title{Sets, Algebra and Logic}
% Basics of statistical processing of biomedical  data
}

% \ifx\pdfoutput\undefined
% % we are running LaTeX, not pdflatex
% \usepackage{graphicx}
% \else
% % we are running pdflatex, so convert .eps files to .pdf
% \usepackage[pdftex]{graphicx}
% \usepackage{epstopdf}
% \fi 
\usepackage{graphics}
\newtheorem{defin}{Definition}[section]
\newtheorem{exa}{Example}[section]
\begin{document}

%%%%%%%%%%%%%%% Define code blocks

% \defverbatim[colored]\factCcode{%
% \begin{lstlisting}[frame=single,language=C]
%   int fact(int n)
%   { int x = 1;
%     while (n > 0)
%      { x = x * n;
%        n = n - 1;
%      }
%     return x;
%   }
% \end{lstlisting}}
% 
% \defverbatim[colored]\factMLcode{%
% \begin{lstlisting}[frame=single]
%   let rec fact n =
%     if n = 0 then 1
%     else n * fact(n - 1);;
% \end{lstlisting}}
% 
% \defverbatim[colored]\badFuncCode{%
% \begin{lstlisting}[frame=single]
%   int rand(void)
%   { static int n = 0;
%     return n = 2147001325 * n + 715136305;
%   }
% \end{lstlisting}}

%%%%%%%%%%%%%%%%%%%%%%

\frame{\titlepage}

% \section*{Biostatistics}

%\subsection{Обсуждаемые темы}


\frame{
  \frametitle{Abstraction}
\Huge
\begin{center}
\begin{tabular}{l|l}
1 & Objects\\
2 & Numbers\\
3 & Arithmetic\\
4 & Algebra\\
5 & Cathegory
\end{tabular}
\end{center}


}

\frame{
  \frametitle{Sets}
\begin{defin}
 A \textbf{set} is a well defined collection of objects
\end{defin}


}

\frame{
  \frametitle{Sets}
\begin{defin}
 A \textbf{set} is a well defined collection of objects
\end{defin}

Naive set theory is not so simple and perfect:
\[
 R = \{ x \mid x \not \in x \} \text{, then } R \in R \iff R \not \in R
\]

}

\frame{
  \frametitle{Membership and subsets}
\Huge
\[
 A = \{1,2,3,4\}, B = \{\alpha, \beta, \gamma\}
\]
\begin{center}
\begin{tabular}{l|l}
$1 \in A$ & $\{1, 4\} \subset A$\\
$\alpha \in B$ & $\{\alpha, \beta\} \nsubseteq A$\\
$\alpha \notin A$ & $\{1,2,3,4\} \subseteq A$ 
\end{tabular}
\end{center}





}
\frame{
  \frametitle{Basic operations}
Union:
\[
 A  \cup B = \{ x: x \in A \,\,\,\textrm{ or }\,\,\, x \in B\}
\]
Intersection:
\[
 A \cap B = \{ x: x \in A \,\land\, x \in B\}
\]
Complement:
\[
 B \smallsetminus A = \{ x\in B \, | \, x \notin A \}
\]


}

\frame{
  \frametitle{Cartesian product and power sets}
\begin{defin}
 \textbf{Power set}  of any set S, written  is the set of all subsets of S, including the empty set and S itself
\end{defin}
\begin{exa}
 \[
  2^{\{1,2,3\}} = \{\{1, 2, 3\}, \{1, 2\}, \{1, 3\}, \{2, 3\}, \{1\}, \{2\}, \{3\}, \varnothing \}
 \]

\end{exa}
\begin{defin}
\textbf{Cartesian product}:
\[
 X\times Y = \{\,(x,y)\mid x\in X \ \text{and} \ y\in Y\,\}
\]
\end{defin}
\begin{exa}
\[
  \{1, 2\} \times \{a, b\} = \{(1, a), (1, b), (2, a), (2, b)\}
\]


\end{exa}


}
\frame{
  \frametitle{Algebraic structure}
Algebraic structure:
\[
 <C, W >
\]
where C -- \textbf{carrier set} and W -- set of operations on C.

\begin{exa}
 \[
  <\mathbb{Z}, \{+, *\}>
 \]

\end{exa}


}



\frame{
  \frametitle{Propositional Logic}
\begin{itemize}
 \item \Large Proposition: statement that is either true or false.
\end{itemize}


}
\frame{
  \frametitle{Propositional Logic}
\begin{itemize}
 \item \Large Proposition: statement that is either true or false.
\item \Huge ``This statement is false.''
\end{itemize}

}
\frame{
  \frametitle{Propositional language}
\begin{itemize}
 \item \Large An infinite set of variables(\textbf{propositions})
\item A set of \textbf{operators}
\item Separate values \textbf{TRUE} and \textbf{FALSE}
\end{itemize}

}
\frame{
  \frametitle{Propositional Operators}
\begin{itemize}
 \item \Large  Negation
\item Disjunction
\item Conjunction
\item Implication
\item Equivalence
\end{itemize}


}
\frame{
  \frametitle{Negation}
\Huge
\begin{center}
\begin{tabular}{l|l}

A & $\neg A$\\ \hline
TRUE & FALSE\\
FALSE & TRUE
\end{tabular}

\end{center}


}
\frame{
  \frametitle{Disjunction (or)}
\Huge

\begin{center}
\begin{tabular}{l|l|l|l}
$A\vee B$& \textbf{TRUE} & \textbf{FALSE} & \textbf{B}\\ \hline
\textbf{TRUE} & TRUE & TRUE\\  \hline
\textbf{FALSE} & TRUE & FALSE \\ \hline
\textbf{A} 
\end{tabular}
\end{center}


}

\frame{
  \frametitle{Conjunction (and)}
\Huge

\begin{center}
\begin{tabular}{l|l|l}
$A\wedge B$& \textbf{TRUE} & \textbf{FALSE} \\ \hline
\textbf{TRUE} & TRUE & FALSE\\ \hline
\textbf{FALSE} & FALSE & FALSE
\end{tabular}
\end{center}


}

\frame{
  \frametitle{Implication (if$\dots$then)}
\Huge

\begin{center}
\begin{tabular}{l|l|l}
$A\Rightarrow B$ & \textbf{TRUE} & \textbf{FALSE} \\ \hline
\textbf{TRUE} & TRUE & FALSE\\ \hline
\textbf{FALSE} & TRUE & TRUE
\end{tabular}
\end{center}


}

\frame{
  \frametitle{Equivalence}
\Huge

\begin{center}
\begin{tabular}{l|l|l}
$A\Leftrightarrow B$ & \textbf{TRUE} & \textbf{FALSE} \\ \hline
\textbf{TRUE} & TRUE & FALSE\\ \hline
\textbf{FALSE} & FALSE & TRUE
\end{tabular}
\end{center}


}

\frame{
  \frametitle{Algebra and Logic}


 \resizebox{1.0\hsize}{!}{$<\{TRUE, FALSE\}, \{\neg,\vee, \wedge, \Rightarrow, \Leftrightarrow \}>$}


}

\frame{
  \frametitle{Inference: Modus Ponens}
\Large Modus Ponens (rule of detachment):


\begin{center}
\begin{tabular}{l|l}
A & Ted is cold\\
$A\Rightarrow B$ & If Ted is cold, he shivers\\ \hline
B & Ted shivers
\end{tabular}
\end{center}


}


\frame{
\frametitle{Dixi}

\begin{center}
\Huge Dixi\end{center}
}
% \frame{
%   \frametitle{Dixi}
% \resizebox{.4\hsize}{!}{ Dixi }
% }
\end{document}
